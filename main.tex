\documentclass[a4paper]{article}

\usepackage[english]{babel}
\usepackage[utf8]{inputenc}
\usepackage{amsmath}
\usepackage{graphicx}
\usepackage[colorinlistoftodos]{todonotes}
\usepackage{float}
\usepackage{hyperref}
\usepackage{listings}

\title{ENGN 8170 Group project\\ Corrosion and Gas Leaking Detection Robot\\ Project Initiation Document}

\author{U6366102, U6561524, U6622423}

\date{Submission date August 9, 2019}

\begin{document}
\pagenumbering{gobble}
\maketitle
\newpage
\tableofcontents
\newpage
\pagenumbering{arabic}
\section{Executive Summary}
The gas leak detection robot has more potential to grow with a variety of sensors and the nature of the environment that is to be dealt with. We propose three potential conceptual designs where, we will be analysing them theoretically and economically in the coming weeks of the semester and finalize on a single design. The concepts are differentiated by the sensor that the robot uses to sense the leak.
\begin{itemize}
    \item {Gas sensor and Pressure sensor based robot where the decision of a leak is decided by the pressure sensor.}
    \item{Infrared thermal imaging to capture the difference in temperature within the working environment.}
    \item{Internet of things} %  Pankhuri please elaborate on this point. 
\end{itemize}
%Sample image code
%begin{figure}[H]
 %  \centering
  % \includegraphics[height = 6cm, width = \textwidth]{}
   %\caption{Original image vs Noisy image}
    %label{fig:S1}
%\end{figure}
\section {Project Background}
 - Can write about Real life scenarios like causes and consequences and this robot will solve the problem.
 We came across numerous concepts on the implementation of gas detection robots. During our literature review we were able to understand the existing methodologies involve detection of gas leak through programmable logic controllers  and supervisory control and data acquisition. The primary reason for not implementing robot based solutions is its uncertainty (detailed in \ref{risks}) which cannot be factored even with thorough system engineering procedures. We are trying to tackle this problem and provide feasible solution which remains profitable for the oil/ gas exporters.  
\section{Expectation}

\section{Work breakdown/Schedule}

\section{Team Organisation}
Ex: Team leader, Mechanical Engineers, System engineers, Electrical engineers, Electronic engineers.
%sample bullet
%begin{itemize}
 %  \item {D0, B, tolerance are passed into the loop}
 %  \item{An atom is selected and their inter dependency with other atoms is computed in an iteration} 
 %  \item{The atom with highest correlation returns a non zero element to X}
  % \item{Residue(r) of OMP gives information about the elements that are uncorrelated with B}
 %  \item{In main loop, Obtain the signal estimate $x_{k}$}
  % \item{Compute the estimate $\hat{b_k}$ with signal estimate $x_{k}$ }
  % \item{Update residue $r_{k+1}$ $\xleftarrow{}$ $r_k - \hat{b_k}$ }
  % \item {Move on to next iteration, until maximum iterations or tolerance values are reached}
%end{itemize}
%newpage
\section {Resources}

\section{Risks}
\label{risks}

\begin{thebibliography}{9}
\bibitem{AEM}
  E. Kreyszig,
  \emph{Advanced Engineering Mathematics}.
   Tenth ed. [Online]
\bibitem{Curve}
  Gurley,
  \emph{Numerical Methods Lecture 5 Curve fitting techniques},
   2001.
\bibitem{MATLAB}
  MathWorks Inc.,
  \emph{MATLab Documentation},
   2019.
\bibitem{Lecnotes}
  Dr.Miaomiao Liu,
  \emph{Lecture Notes Semester 1},
   2019.
\bibitem{data}
Y. Xu and W. Yin. \emph{A fast patch-dictionary method for whole-image recovery.} UCLA CAM report 13-38, 2013.
\end{thebibliography}
\end{document}