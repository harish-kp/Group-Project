\documentclass[a4paper]{article}

\usepackage[english]{babel}
\usepackage[utf8]{inputenc}
\usepackage{amsmath}
\usepackage{graphicx}
\usepackage[colorinlistoftodos]{todonotes}
\usepackage{float}
\usepackage{hyperref}
\usepackage{listings}

\title{ENGN 8170 Group project\\ Corrosion and Gas Leaking Detection Robot\\ Project Initiation Document}

\author{U6366102, U6561524, U6622423, U6647679, U6471573}

\date{Submission date August 9, 2019}

\begin{document}
\pagenumbering{gobble}
\maketitle
\newpage
\tableofcontents
\newpage
\pagenumbering{arabic}
% visit https://github.com/harish-kp/Group-Project for detailed commits and updates
\section{Executive Summary}
The gas leak detection robot has more potential to grow with a variety of sensors and the nature of the environment that is to be dealt with. We propose three potential conceptual designs where, we will be analysing them theoretically and economically in the coming weeks of the semester and finalize on a single design. The concepts are differentiated by the sensor that the robot uses to sense the leak.
\begin{itemize}
    \item {Gas sensor and Pressure sensor based robot where the decision of a leak is decided by the pressure sensor.}
    \item{Infrared thermal imaging to capture the difference in temperature within the working environment.}
    \item{Internet of things} %  Pankhuri please elaborate on this point. 
\end{itemize}
%Sample image code
%begin{figure}[H]
 %  \centering
  % \includegraphics[width =8.5cm \textwidth]{}
   %\caption{Original image vs Noisy image}
    %label{fig:S1}
%\end{figure}
\section{Stakeholders}
In the following sub-sections, we will be identifying and listing all our stakeholders, priority, requirements for the pipeline inspection robot and their usage of the robot.
\begin{figure}[htbp]
    \centerline{\includegraphics[width=8.5cm]{Stakeholders.png}}
    \caption{List of Stakeholders}
    \label{fig:2.1}
\end{figure}
\subsection{List of Stakeholders}
The table 2.1 lists the stakeholders of the project, description of the stakeholders, interest, power and strategy used to manage them.

\begin{tabular}{ |l|l|l|l|l|l|  }
 \hline
 U.I.D     & Name of the stakeholder &Description&Interest&Power&Strategy\\
 \hline
 S-01   & Investors    &&&&\\
 S-02   &Suppliers  &&&&\\
 S-03 &Producion/Exporters &&&&\\
 S-04    &Maintenance authorities &&&&\\
S-05&   Engineers  &&&&\\
 S-06& Governance/Deciding Authorities &&&&\\
 S-07& Customers&&&&\\
 \hline
\end{tabular}
\section {Project Background}
 - Can write about Real life scenarios like causes and consequences and this robot will solve the problem.\bigskip \\ 
 We came across numerous concepts on the implementation of gas detection robots. During our literature review we were able to understand the existing methodologies involve detection of gas leak through programmable logic controllers  and supervisory control and data acquisition. The primary reason for not implementing robot based solutions is its uncertainty (detailed in \ref{risks}) which cannot be factored even with thorough system engineering procedures. We are trying to tackle this problem and provide feasible solution which remains profitable for the oil/ gas exporters.  
\section{Expectation}

\section{Work breakdown/Schedule}

\section{Team Organisation}
Ex: Team leader, Mechanical Engineers, System engineers, Electrical engineers, Electronic engineers.
%sample bullet
%begin{itemize}
 %  \item {D0, B, tolerance are passed into the loop}
 %  \item{An atom is selected and their inter dependency with other atoms is computed in an iteration} 
 %  \item{The atom with highest correlation returns a non zero element to X}
  % \item{Residue(r) of OMP gives information about the elements that are uncorrelated with B}
 %  \item{In main loop, Obtain the signal estimate $x_{k}$}
  % \item{Compute the estimate $\hat{b_k}$ with signal estimate $x_{k}$ }
  % \item{Update residue $r_{k+1}$ $\xleftarrow{}$ $r_k - \hat{b_k}$ }
  % \item {Move on to next iteration, until maximum iterations or tolerance values are reached}
%end{itemize}
%newpage
\section {Resources}

\section{Risks}
\label{risks}
The potential risk of the projects are being listed below
\begin{itemize}
\subsection{Engineering risks}
    \item{Robot might breakdown inside the pipe, at a section which is inaccessible from the outside.}
    \item {During operation, robot can restrict the flow-rate of the fluid.}
    \item{Multiple leaks located very close to each other can be mistaken as a single leakage point.}
    \item{Minute fracture in pipe might not be detected due to sensor noise.}
    \item{Robot performs well under test conditions, but not might fail under actual (unprecedented) condition.}
    \item{Communication stopped due to failure of bluetooth/wifi module or signals blocked due to underground pipe.}
    \item{Gas can leak inside robot’s hull, corrupting the controller}
    \item{Electrical discharge in robot’s wiring can cause an explosion in the pipe (flammable gas).}
    \item{Unpredictable number of iterations of prototyping and testing can result in deadlines not met.}
    \item{Failure to track exact location of robot can lead to an inaccurate detection of leakage location.}
\subsection{Financial risks}
    \item{Lack of funds for fabrication of robot.}
\subsection{Administrative risks}
    \item{Availability of all team members during all stages of project development cannot be ensured.}
    \item{Delay in delivery of important components (eg. Sensors) can delay the entire project.}
    \item{Lack of appropriate guidance while facing any unidentified errors during software operation can cause delays.}
    \item{ANU maker-space lab might not be available as per our requirement, thus delaying fabrication.}

\end{itemize}
\begin{thebibliography}{9}
\bibitem{AEM}
  E. Kreyszig,
  \emph{Advanced Engineering Mathematics}.
   Tenth ed. [Online]
\bibitem{Curve}
  Gurley,
  \emph{Numerical Methods Lecture 5 Curve fitting techniques},
   2001.
\bibitem{MATLAB}
  MathWorks Inc.,
  \emph{MATLab Documentation},
   2019.
\bibitem{Lecnotes}
  Dr.Miaomiao Liu,
  \emph{Lecture Notes Semester 1},
   2019.
\bibitem{data}
Y. Xu and W. Yin. \emph{A fast patch-dictionary method for whole-image recovery.} UCLA CAM report 13-38, 2013.
\end{thebibliography}
\end{document}